% Created 2018-06-05 Tue 15:32
\documentclass[11pt]{article}
\usepackage[utf8]{inputenc}
\usepackage[T1]{fontenc}
\usepackage{fixltx2e}
\usepackage{graphicx}
\usepackage{longtable}
\usepackage{float}
\usepackage{wrapfig}
\usepackage{rotating}
\usepackage[normalem]{ulem}
\usepackage{amsmath}
\usepackage{textcomp}
\usepackage{marvosym}
\usepackage{wasysym}
\usepackage{amssymb}
\usepackage{hyperref}
\tolerance=1000
\author{Nicholas Yang}
\date{\today}
\title{Ruby Source Code Notes}
\hypersetup{
  pdfkeywords={},
  pdfsubject={},
  pdfcreator={Emacs 25.2.1 (Org mode 8.2.10)}}
\begin{document}

\maketitle
\tableofcontents


\section{Intro}
\label{sec-1}
These are my notes for the Ruby source code. I hope to document my
attempts to understand the Ruby code.

\section{Globals}
\label{sec-2}
\begin{itemize}
\item \label{Qundef} \verb~Qundef~ --- (?) A magic number that represents undefined
\end{itemize}

\section{Helper Functions/Macros}
\label{sec-3}
\begin{itemize}
\item \label{has_delayed_token} \verb~has_delayed_token~ --- For Ruby parser,
always false
\end{itemize}

\section{Types}
\label{sec-4}
\begin{itemize}
\item \label{parser_params} \verb~parser_params~ --- Oh boy, this is a complicated
one. Consists of a massive struct that contains:
\begin{itemize}
\item A \hyperref[rb_imemo_tmpbuf_t]{\verb~rb\_imemo\_tmpbuf\_t~} pointer called heap (so
I'm guessing this is just the heap/free list for the parser?).
\item A \hyperref[YYSTYPE]{\verb~YYSTYPE~} pointer called \verb~lval~
\item A struct called \texttt{\verb~lex~}
\item Two \verb~stack_type~, i.e. \texttt{\verb~VALUE~}'s called \verb~cond_stack~ and
\verb~cmdarg_stack~
\item A bunch of \verb~int~'s:
\begin{itemize}
\item \verb~tokidx~
\item \verb~toksiz~
\item \verb~tokline~
\item \verb~heredoc_end~
\item \verb~heredoc_indent~
\item \verb~heredoc_line_indent~
\item \verb~line_count~
\item \verb~ruby_sourceline~
\end{itemize}
\end{itemize}
\end{itemize}


\begin{itemize}
\item \label{VALUE} \verb~VALUE~ --- \verb~uintptr_t~
\item \label{yytokentype} \verb~yytokentype~
\item \label{YYSTYPE} \verb~YYSTYPE~ ---
\item \label{rb_imemo_tmpbuf_t} \verb~rb_imemo_tmpbuf_t~ --- A linked list with a payload consisting of
two \url{file:///Users/nicholas/cs_projects/gsoc_ruby_types/~VALUE~}'s: \verb~flags~ and \verb~reserved~ and a \verb~VALUE~ pointer
called\ldots{}.\textasciitilde{}ptr\textasciitilde{} which apparently is a malloc'd buffer. Also has a
\verb~size_t~ called \verb~cnt~ which is the size of that buffer. Finally has
a pointer to another \verb~rb_imemo_tmpbuf_struct~ called \verb~next~.
\end{itemize}

\section{Lexer}
\label{sec-5}

\begin{itemize}
\item \label{yylex} \verb~yylex~ --- A wrapper for \hyperref[parser_yylex]{\verb~parser\_yylex~} with some extra
options. Initializes a \hyperref[yytokentype]{\verb~yytokentype~}. Assigns \verb~p~, a struct of
\url{file:///Users/nicholas/cs_projects/gsoc_ruby_types/~parser\_params~} an \texttt{\verb~lval~} (with type \hyperref[YYSTYPE]{\verb~YYSTYPE~}) and gives \verb~lval~ a
\verb~val~ of \hyperref[Qundef]{\verb~Qundef~}. (I guess initializing \verb~p~ with an empty \verb~lval~?)
Then calls \hyperref[parser_yylex]{\verb~parser\_yylex~} with \verb~p~ and assigns the result to the
\verb~yytokentype~ variable. If \hyperref[has_delayed_token]{\verb~has\_delayed\_token~} (never in Ruby
parser) then call \verb~dispatch_delayed_token~. Else if \verb~parser\_yylex~
returns 0 then call \verb~dispatch_scan_event~, which, uh, in Ruby parser
does nothing (well, it casts 0 to void).
\item \label{parser_yylex} \verb~parser_yylex~ ---
\end{itemize}

\section{Terminals}
\label{sec-6}
\begin{itemize}
\item \label{kALIAS} --- \verb~alias~
\item \label{tGVAR} --- Global variable
\item \label{tBACK_REF} --- \verb~$+~, \verb~$&~, \verb~$`~, \verb~$\~
\end{itemize}
\section{Grammar Rules}
\label{sec-7}
\begin{itemize}
\item \verb~program~ --- Just the top level rule. Alias for \texttt{\verb~top\_compstmt~}
\item \label{top_compstmt} \verb~top_compstmt~ --- Again just \hyperref[top_stmts]{\verb~top\_stmts~} with an
\hyperref[opt_terms]{\verb~opt\_terms~} add on.
\item \label{top_stmts} --- Either nothing, one single \url{file:///Users/nicholas/cs_projects/gsoc_ruby_types/~top_stmt~} or a
\item \label{opt_terms}
\end{itemize}

\section{Sources}
\label{sec-8}

\begin{itemize}
\item \url{https://whitequark.org/blog/2013/04/01/ruby-hacking-guide-ch-11-finite-state-lexer/}
\item \url{https://ruby-hacking-guide.github.io/}
\end{itemize}
% Emacs 25.2.1 (Org mode 8.2.10)
\end{document}